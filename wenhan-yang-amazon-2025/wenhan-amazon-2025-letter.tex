Dear Members of the Amazon Fellowship Selection Committee:

\textbf{I write with enthusiasm to recommend Wenhan Yang for an Amazon PhD
Fellowship}. I have had the privilege of working closely with Wenhan in my
capacity as the instructor for CS 239: Introduction to Human-Computer
Interaction, a highly interactive, group project-based course for graduate
students. His commitment to (i) \textbf{creative problem solving} and
(ii) \textbf{active collaboration} sets him apart.

In his research, Wenhan brings a \textbf{creativity to tackling complex real-world
problems for improving the robustness of vision-language foundation models, such
as CLIP and LLaVA}. His research on RoCLIP mitigates safety and reliability
issues at training by relying on semantic similarities between captions and
images to captions. Wenhan's approach protects against a threat model where
incorrect captions are intentionally introduced. Since publishing RoCLIP at
NeurIPS 2023, Wenhan has expanded his semantic approach in follow-up
work on SafeCLIP (ICML 2024) and ongoing work. 

In the classroom, I have seen how Wenhan \textbf{brings this creativity to
address the real-world need for learning a new language}. With his team, Wenhan
conducted user research by interviewing 10 and surveying an additional 35 adults
about their experiences learning new languages. They found that a key barrier to
learning languages is self-consciousness about practicing a new language, which
means people do not practice, only to become more self-conscious, leading to a
feedback loop. Based on this insight, Wenhan and his team have been developing a
chat-bot for practicing scenarios and role playing in English, Chinese, and
Japanese. Wenhan has been leading the features interfacing with LLMs using text
and speech modalities. Through the process, Wenhan has demonstrated his
\textbf{ability to connect and ground ML problems with experiences that users
face}. In other words, \textbf{Wenhan can not only motivate and explain ML
problems with theory but also with real-world applications with users}.

Wenhan regularly attends Design Studio, open collaborative sessions for the
course, where his team has become a fixture every Friday afternoon. In this
setting, I have observed how Wenhan can inject a sense of fun and humor even
under tight deadlines. One thing that stands out to me is Wenhan's demonstrated
capacity to \textbf{engage productively with students in his group who have different technical backgrounds},
including those who are less interested in ML and new to HCI. 
\textbf{I have asked Wenhan to serve as a teaching assistant for my
undergraduate human-computer interaction course next quarter}. I am excited to
see how Wenhan continues to apply his creative problem solving and collaborative
mindset to teaching and becoming a role model for students. 

\textbf{I
enthusiastically support Wenhan's application for the Amazon Fellowship}. 
Please do not hesitate to contact me at \href{mailto:emjun@cs.ucla.edu}{emjun@cs.ucla.edu} if I can provide any additional information.